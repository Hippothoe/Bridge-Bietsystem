\chapter{Eröffnungen}

% -------------  Eröffnungen  -------------

\section{Bedeutungen der Eröffnungen} \label{sec:Bedeutungen der Eröffnungen}

\begin{description}
    \item{\hyperref[sec:Antworten auf 1T]{1\t} \tabto{5mm} a. 12-14 balanced, ohne 5er OF (mind. 2er\t, \kl kann länger sein) \newline
              \tabto{5mm} b. 11-22 Einfärber/Zweifärber, längste Farbe \tl \newline
              \tabto{5mm} c. 12-22 4414}
    \item{\hyperref[sec:Antworten auf 1K]{1\k} \tabto{5mm} a. 18-20 balanced, ohne 5er OF (mind. 2er\k, \tl kann länger sein) \newline
              \tabto{5mm} b. 11-22 Einfärber/Zweifärber, längste Farbe \k \newline
              \tabto{5mm} c. 12-22 4441, 4141, 1444
              \tabto{5mm} d. 11-15 5er\tl 4er\kl (außer mit schwachen \k s oder sehr starken \t s)}
    \item{\hyperref[sec:Antworten auf 1OF]{1OF} \tabto{5mm} a. 12-14 balanced, 5er OF \newline
              \tabto{5mm} b. 18-20 balanced, 5er OF \newline
              \tabto{5mm} b. 11-22 Einfärber/Zweifärber, längste Farbe}
    \item{\hyperref[sec:Antworten auf 1SA]{1SA} \tabto{5mm} 15-17 balanced, kann 5er OF beinhalten}
    \item{\hyperref[sec:Antworten auf 2T]{2\t*} \tabto{5mm} a. 25-26 balanced, kann 5er OF beinhalten \newline
              \tabto{5mm} b. 27+ balanced ohne 5er OF \newline
              \tabto{5mm} c. OF Einfärber mit mind. 8,5 Spielstichen
              \tabto{5mm} d. UF Einfärber mit mind. 9,5 Spielstichen}
    \item{\hyperref[sec:Antworten auf 2K]{2\k*} \tabto{5mm} a. pre-empt, 6+ OF \newline
              \tabto{5mm} b. 23-24 balanced, kann 5er OF beinhalten}
    \item{\hyperref[sec:Antworten auf 2C]{2\c*} \tabto{5mm} pre-empt, 5+\c 5+UF}
    \item{\hyperref[sec:Antworten auf 2P]{2\p*} \tabto{5mm} pre-empt, 5+\p 5+UF}
    \item{\hyperref[sec:Antworten auf 2SA]{2SA} \tabto{5mm} 21-22 balanced, kann 5er OF beinhalten}
\end{description}

% -------------  Generelle Prinzipien  -------------

\section{Generelle Prinzipien} \label{sec:Generelle Prizipien}
\subsection{20er-Regel}
Üblicherweise eröffnet man 1 in Farbe ab 12 F. Eine schwächere Hand kann auch mit 1 in Farbe eröffnet werden, wenn die \textbf{20er-Regel} erfüllt ist: Anzahl Figurenpunkte + Anzahl Karten in längsten beiden Farben $\geq 20$. Die Figuren sollten dabei hauptsächlich in den langen Farben konzentriert sein.

\subsection{Wahl der Eröffnungsfarbe}

Bei Zweifärbern eröffnet man die längere Farbe. Bei zwei gleich langen Farben eröffnet man i.d.R. die höhere. Mit 16+ Punkten eröffnet man stattdessen von zwei gleichlangen Farben die niedrige und reizt die höhere im Rebid als Reverse.

Eine Ausnahme sind Zweifärber mit 5er\tl und 4er\k. Falls die Hand sich für eine Reverse-Reizung eignet, sollte dies getan werden. Falls nicht (11 - 15 Punkte) lohnt es sich oft, trotz nur 4er\kl die Hand mit 1\kl zu eröffnen, um im Rebid mit 2\tl einen schwachen Zweifärber zeigen zu können. Dies sollte man unterlassen, wenn die \t s sehr schön oder die \k s sehr schlecht sind.

\subsection{Eröffnung in 3. Hand}
In dritter Hand kann, um es dem Gegner in 4. Hand schwerer zu machen, 1 in Farbe schon ab 10 F eröffnet werden, idealerweise mit 5er-Länge und konzentrierten Figuren.

Pre-empts in dritter Hand können weniger Karten in der gereizten Farbe enthalten, als versprochen wurde.

In jedem Fall ist auf die Gefahrenlage zu achten.

\subsection{Eröffnung in 4. Hand}
In 4. Hand ist es mit 10-12 F wahrscheinlich, dass beide Parteien ungefähr gleich stark sind. Daher ist die Seite mit Pik-Fit im Vorteil. Eröffnung nach der \textbf{15er-Regel}: Wenn Anzahl Figurenpunkte + Anzahl Piks $\geq 15$, dann eröffnen, sonst passen. Mit 13 und mehr F auch ohne Piks eröffnen, da die Hand dann entsprechende Verteilungsstärke besitzt.

TODO Eröffnungen in 4. Hand, die sonst Sperrgebote wären

\pagebreak

% -------------  Antworten auf 1T  -------------

\section{Antworten auf 1\tl(mit Transfers)} \label{sec:Antworten auf 1T}

\begin{description}
    \item{Pass \nf 0-5 F (viele 3-5 F Hände eignen sich aber für Transfers!)}
    \item{1\k* \f 3+ F, 4+\c \ra \ref{subsec:1T 1K}
          \tabto{12mm} mit 3-5 F sollte man 5er\cl oder min. 43 in \k\pl haben
          \tabto{12mm} mit gf und nur 4er\cl stattdessen 2\tl reizen}
    \item{1\c* \f 3+ F, 4+\p
          \tabto{12mm} mit 3-5 F sollte man 5er\pl oder min. 43 in \k\cl haben
          \tabto{12mm} mit gf und nur 4er\pl stattdessen 2\tl reizen}
    \item{1\p* \f a. 6-7 balanced, keine 4+ OF
          \tabto{12mm} b. 6-11 unbalanced, keine 4+ OF, Notgebot}
    \item{1SA \nf 8-11 balanced, keine 4+ OF}
    \item{2\t* \gf Relay, keine 5+ OF, 4er OF möglich}
    \item{2\k* \f 3-11; 5+\pl 4+\c}
    \item{2OF \nf 3-8; 6+OF}
    \item{2SA \nf 11-12; 4333 ohne 4er OF} \ra 3\tl to play; Rest gf
    \item{3UF* \nf 9-11; 6+UF (mind. KB10xxx), ohne 4er OF} \ra alles außer pass ist gf
    \item{3OF \nf 3-6; 7+OF, Weak Jump}
    \item{3SA \nf 13-15; 4333 Hand ohne 4er OF}
\end{description}

Prioritäten bei den Antworten:
\begin{itemize}
    \item{Mit gf Händen ohne 5er OF reizt man 2\t*}
    \item{Mit 44 in den Oberfarben ohne gf reizt man 1\k*}
    \item{Mit 5+\p 4+\cl antwortet man ohne gf 2\k*, mit gf 1\c*}
\end{itemize}

% -------------  1T 1K*  -------------
\newpage

\subsection{1\tl 1\k*} \label{subsec:1T 1K}

\begin{description}
    \item{Pass \tabto{12mm} verboten}
\end{description}
Mit der balanced Hand führt man den Transfer immer aus (auch mit \c-Fit):
\begin{description}
    \item{1\c* \nf bestätigt 12-14 balanced, 2er bis 4er\c \ra \ref{subsubsec:1T 1K 1C}}
\end{description}
Mit unbalanced Händen ohne Fit beschreibt man seine Hand weiter:
\begin{description}
    \item{1\p \nf 11-22; Zweifärber 5+\tl 4+\p, fast nie zu passen \ra \ref{subsubsec:1T 1K 1P}}
    \item{1SA* \f 16+; starker Einfärber mit 6+\t} \ra \ref{subsubsec:1T 1K 1SA}
    \item{2\t \nf a. 11-15; schwächerer Zweifärber 5+\tl 4+\k 
              \tabto{12mm} b. 11-15; schwächere Einfärber 6+\t}
    \item{2\k \nf 16-22; starker Zweifärber 5+\tl 4+\k, fast nie zu passen}
    \item{2\c \nf 11-15; 3+\c (3er fit kann nur 1345 sein, meist 4er, eher unbalanced)}
    \item{2\p \nf 10+P; Zweifärber 6+\tl 5+\p}
    \item{3\t \nf 13-15; 7+\tl} \ra alles außer pass ist gf, neue Farben sind Stopper
    \item{3SA \nf to play, Eröffner hat semi-forcing in \tl mit \c-Kürze}
    \item{4\c \nf 10-12; 6+\tl5\c}
\end{description}
Mit unbalanced Händen und Fit:
\begin{description}
    \item{2\c \nf 11-15; 3+\c (3er fit kann nur 1345 sein, meist 4er, eher unbalanced)}
    \item{2SA* \f 19+ ; 4+\c }
    \item{3\k* \f 19+; 4+\c, Splinter}
    \item{3\c \nf 16-18; 4+\c}
    \item{3\p* \f 19+; 4+\c, Splinter}
    \item{4\t \f SI; 6+\tl4\c}
\end{description}

% -------------  1T 1K 1C*  -------------

\subsubsection{1\t1\k1\c*} \label{subsubsec:1T 1K 1C}
Eröffner ist balanced mit 12-14, 2er bis 4er\c, hat den Transfer ausgeführt.
Partners Hand ist noch ein Myterium, man weiß nur 3+P und 4+\c.

Ohne Vollspielinteresse sind die möglichen Endkontrakte jetzt 1\c/1\p/1SA/2\k/2\c/3\t:

\begin{description}
    \item{Pass \nf erlaubt und gewünscht mit $\leq$5P}
    \item{1\p \nf $<$inv, 4+\c4\p \ra natürlich }
    \item{1SA \nf to play \ra 2\c (hatte 4er\c)}
    \item{2\t* \f max inv Relay (Two Way Checkback) \ra \ref{subsubsec:1T 1K 1C 2T}}
    \item{2SA* \f EÖ muss jetzt 3\t reizen \ra alles außer passe ist SI-\tl (5++\t)}
\end{description}

Über folgende Gebote kann man zum Vollspiel einladen:

\begin{description}
    \item{2\t* \f max inv Relay (Two Way Checkback) \ra \ref{subsubsec:1T 1K 1C 2T}}
    \item{2\c \nf Mini-INV mit 5er\cl (nur INV falls EÖ 4er\cl hat)}
    \item{2\p \nf INV mit 4\c4\p}
    \item{3UF \nf INV 5\c5UF \ra pass verneint 3er\c}
\end{description}

Mit gf+ stehen folgende Gebote zu Verfügung:

\begin{description}
    \item{2\k* \f beliebiges gf mit 5er\c}
    \item{2SA* \f EÖ muss jetzt 3\t reizen \ra alles außer passe ist SI-\tl (5++\t)}
    \item{3\c \gf 6+\c}
    \item{3\p* \gf 6+\c, Splinter}
    \item{3SA \nf to play}
    \item{4UF \gf 6+\c, Splinter}
    \item{4\c \nf to play}
\end{description}

% -------------  1T 1K 1C 2T* -------------

\subsubsection{1\t1\k1\c2\t* (Two Way Checkback)} \label{subsubsec:1T 1K 1C 2T}

Mit schwachen Händen kann man jetzt in 2\kl oder 2\cl landen. Mit einladenden Händen kann man seine Einladung jetzt genauer beschreiben.

\begin{description}
    \item{2\k \nf kein 4er\c}
    \item{\tabto{5mm} \ra \tabto{12mm} Pass \tabto{22mm} will jetzt 2\kl spielen}
    \item{\tabto{12mm}2\c \tabto{22mm} INV mit 5er\c}
    \item{\tabto{12mm}2\p \tabto{22mm} INV mit 5\c4\p}
    \item{\tabto{12mm}2SA \tabto{22mm} INV mit 4+\c}
    \item{\tabto{12mm}3UF \tabto{22mm} INV mit 4\c5+UF}
    \item{\tabto{12mm}3\c \tabto{22mm} 9-11 mit 6+\c}
    \item{2\c \nf 4er\cl \ra natürlich}
\end{description}

% -------------  1T 1K 1P -------------

\subsubsection{1\t1\k1\p} \label{subsubsec:1T 1K 1P}

Eröffner hat Zweifärber 5+\tl5+\p, Partner hat 4+\cl und muss sich jetzt überlegen, wohin in diesem Leben.

Mit nur 4-5\cl und weder \pl noch \tl sind mögliche Endkontrakte noch 1\p/1SA/2SA/3SA. 

\begin{description}
	\item{Pass \nf sei erlaubt mit weniger als 6P}
	\item{1SA \nf 6-10 natürlich}
	\item 2\k* \gf 4.Farbe forcing, hier gameforcing (5+\cl da nicht zu Beginn schon 1\t-2\t)
	\item 2SA \nf invite zu 3SA
\end{description}

Partner kann mit besonderer Länge auch auf \cl insistieren:

\begin{description}
	\item 2\c \nf 9-11 mit 6+\c
	\item 3\k \gf Werte mit 5\cl und 5\k
	\item 3\c \gf Werte mit 6+\c
	\item 4\c \gf knappes gf mit 7+\c
\end{description}

Hat Partner auch \p, so bietet sich natürlich an:

\begin{description}
	\item 2\k* \gf 4.Farbe forcing, hier gameforcing (5+\cl da nicht zu Beginn schon 1\t-2\t) 
	\item 2\p \nf 6-10 mit 4er-Fit in \p
	\item 3\p \nf INV; 4+\p
	\item 4m* \f Splinter mit \p-Fit (auch 4\t ist hier!)
	\item 4\p \nf Spielvorschlag (max 1 Keycard ohne Trumpfdame, sonst über 2\k gehen)
\end{description}

Hat Partner \t, so gibt es:

\begin{description}
	\item 2\t \nf 3-9 ausbessernd \ra 2\cl zeigt 3er und invite
	\item 2\k* \gf 4.Farbe forcing, hier gameforcing (5+\cl da nicht zu Beginn schon 1\t-2\t) 
	\item 3\t \nf invite mit \t-Fit
\end{description}
Vorsicht: 4\tl ist \t-Fit und Splinter!

%--------------  1T 1K 1SA -----------

\subsubsection{1\t1\k1SA} \label{subsubsec:1T 1K 1SA}

Eröffner hat Einfärber mit 6+\tl und 16+ FL, Partner hat 4+\c.

Ist Partner schwach, so ist 2\tl ein guter Kontrakt, außer Eröffner zeigt noch weitere Stärke. In letzterem Fall ist aus 3\t, 5\t, 3SA, 4\cl zu wählen:
\begin{description}
	\item Pass \nf verboten
	\item 2\t* \nf 3-7, kann kurz in \tl sein
	\rarelayitem Pass \tab erlaubt mit 16-17
	\relayitem 2\k* \tab 21+ gf \ra natürlich
	\relayitem 2\c \tab 18-20 6+\tl3\cl nf. \ra 3\tl= to play, Rest = gf
	\relayitem 2\p \tab frei
	\relayitem 2SA \tab 18-20 mit Stoppern \ra 3\tl= to play
	\relayitem 3\t \tab 18-20 \ra alles außer pass = gf 
\end{description}

Mit mehr Stärke aber ohne Interesse an \tl kann Partner SA vorschlagen oder auf \cl insistieren:

\begin{description}
	\item 2\k* \gf gameforcing Relay;
	\rarelayitem 2\c \tab 3er\c
	\relayitem Rest \tab kein 3er\c, so natürlich wie möglich
	\item 2\c \gf 9+ mit 6+\c, jetzt gameforcing
	\item 2SA \nf invite zu 3SA
	\item 3\k \gf 5+\cl5+\k
	\item 3\c \gf SI, sehr gute 6+\c
	\item 3SA \nf Spielvorschlag
	\item 4\c \gf knappes gf mit 7+\c
\end{description}

Mit \t-Fit hat Partner folgende Möglichkeiten:
\begin{description}
	\item 2\t* \nf 3-7, kann auch kurz in \tl sein, siehe oben
	\item 2\k* \gf gameforcing Relay, siehe oben
	\item 3\t \nf invite, \t-Fit
	\item 4\t \nf bed. RKCB-\t
	\item 4\k* \gf Splinter; \t-Fit
\end{description}


% -------------  1T 1C*  -------------

\subsection{1\tl1\c} \label{subsec:1T 1C}


%--------------  1T 1C* 1SA ----------


% -------------  1T 1P*  -------------

\subsection{1\tl 1\p} \label{subsec:1T 1P}

% -------------  1T 1SA  -------------

\subsection{1\tl 1SA} \label{subsec:1T 1SA}

% -------------  1T 2T*  -------------

\subsection{1\tl 2\t} \label{subsec:1T 2T}

% -------------  1T 2K*  -------------

\subsection{1\tl 2\k} \label{subsec:1T 2K}

% -------------  Antworten auf 1K  -------------

\section{Antworten auf 1\k} \label{sec:Antworten auf 1K}

\begin{description}
    \item{Pass \nf 0-5 F (mit 3-5 F und 5er OF muss nicht gepasst werden)}
    \item{1\c \f 3+ F, 4+\c
          \tabto{12mm} mit 3-5 F sollte man 5er\cl haben
          \tabto{12mm} mit gf und nur 4er\cl stattdessen 2\tl reizen}
    \item{1\p \f 3+ F, 4+\p
          \tabto{12mm} mit 3-5 F sollte man 5er\pl haben
          \tabto{12 mm} mit gf und nur 4er\pl stattdessen 2\tl reizen}
    \item{1SA \nf 8-11 balanced oder unbalanced Notgebot, keine 4+ OF}
    \item{2\t* \gf Relay, keine 5+ OF, 4er OF möglich, \ra\ref{subsec:1K 2T}}
    \item{2\k* \f 3-11; 5+\p 4+\c}
    \item{2\c \nf 3-8; 6+\c}
    \item{2\p \nf 3-8; 6+\p}
    \item{2SA \nf 11-12; 4333 ohne 4er OF} \ra 3\kl to play; Rest pf
    \item{3\t \nf 9-11; 6+\tl (mind. KB10xxx), ohne 4er OF} \ra alles außer pass ist gf
    \item{3\k \nf 9-11; 6+\kl (mind. KB10xxx), ohne 4er OF} \ra alles außer pass ist gf
    \item{3\c \nf 3-6; 7+\c, Weak Jump}
    \item{3\p \nf 3-6; 7+\p, Weak Jump}
    \item{3SA \nf 13-15; 4333 Hand ohne 4er OF}
\end{description}

Prioritäten bei den Antworten:
\begin{itemize}
    \item{Mit gf Händen ohne 5er OF reizt man 2\t*}
    \item{Mit 44 in den Oberfarben ohne gf reizt man 1\c}
    \item{Mit 5+\p 4+\c antwortet man ohne gf 2\k*, mit gf 1\p}
\end{itemize}

% -------------  1K - 2T*  -------------

\subsection{1\kl2\t} \label{subsec:1K 2T}
Die Reizung ist partieforcierend.
\begin{description}
	\item 2\k* \f 18-20 balanced
	\item 2\c \f 11+; 5+\kl4+\cl oder 4441/1444
	\item 2\p \f 11+; 5+\kl4+\pl oder 4144
	\item 2SA* \f 11+; 5+\kl4+\t 
	\item 3\t* \f 11+; 6+\kl Einfärber
	\item 3\k \f 16+; 6+\kl starker Einfärber
\end{description}

\subsubsection{1\k2\t2\k}
Partner hat keine 5er OF, aber kann hier nun 4er zeigen. 
\begin{description}
	\item Pass \nf verboten
	\item 2\c \f 4er\cl (evtl. auch 4er\p)
	\item 2\p \f 4er\p (kein 4er\c)
	\item 2SA \f UF-Stayman
	\rarelayitem 3m \tab 4+m
	\relayitem 3SA \tab keine 4+UF
	\item 3m \f 5+m
\end{description}
Eröffner bestätigt Fit durch Wiedergebot oder verneint mit SA.

\subsubsection{1\k2\t2SA}
\mat{Ein Oberfarbenfit ist unwahrscheinlich, daher zeigen diese Gebote Kontrollen. 
\begin{description}
	\item 3\t/\k \f Fitbestätigung, potentiell ungeeignet für 3SA
	\item 3M* \f kein UF-Fit, Kontrolle in M 
	\item 3SA \nf abschließend
	\item 4\t/\k \f bedingte RKCB, Fitbestätigung und Schlemminteresse
\end{description}}




% -------------  Antworten auf 1OF  -------------

\section{Antworten auf 1\c/1\p} \label{sec:Antworten auf 1OF}

Bei Antworten auf 1 in OF kann ein 4er-Support ähnlich zu \textbf{Bergen Raises} mit verschiedenen Antworten je nach Stärke auf der 3er-Stufe gezeigt werde (bzw. ab 16 FV mit 2SA). Doppelsprünge in ungereizten Farben sind Splinter und versprechen ebenfalls 4er-Support sowie 14-15 FV. Stärke Splinter werden über 2SA gereizt.
Ohne 4er-Support kann spielforcierend nach dem Prinzip \textbf{2 over 1 gameforcing} gereizt werden.

% -------------  Antworten auf 1C  -------------

Antworten auf 1\c(größtenteils analog für 1\p):
\begin{description}
    \item{1\p: 4+\p, ab 6 FL \ra s. \ref{subsec:1c 1p}}
    \item{1SA*: Notgebot, 6-11 FL, 3+\c mit 6-8 FV möglich \ra s. \ref{subsec:1c 1SA}}
    \item{2\t*: 2+\t, spielforcierend \ra s. \ref{subsec:1c 2t} \newline (von der gepassten Hand natürlich, nf)}
    \item{2\k*: 5+\k, spielforcierend \newline (von der gepassten Hand natürlich, nf)}
    \item{2\c*: 3er\c, 9-12 FV \ra s. \ref{subsec:1c 2c}}
    \item{2\p: 6er\p, 5-8 F mit Mittelwerten \ra s. \ref{subsec:1c 2p}}
    \item{2SA*: 4+\c, ab 16 FV \ra s. \ref{subsec:1c 2SA} \newline (von der gepassten Hand 11 FL, einladend zu 3SA, nf)}
    \item{3\t*: 4+\c, 9-11 FV oder 14-15 FV \ra s. \ref{subsec:1c 3t}}
    \item{3\k*: 4+\c, 12-13 FV, einladend \ra s. \ref{subsec:1c 3k}}
    \item{3\c: 4+\c, 5-8 FV}
    \item{3SA: natürlich}
    \item{3\p*/4\t*/4\k*: 4+\c, Kürze \p/\t/\k, 14-15 FV (Splinter)}
    \item{4\c: meistens 5er\c, wenig Figurenpunkte}
\end{description}

% -------------  Antworten auf 1P  -------------

Unterschiede nach Eröffnung 1\p
\begin{description}
    \item{2\c: 5+\c, spielforcierend}
    \item{3\c: 7er\c, 5-8 F mit Mittelwerten}
    \item{4\c*: 4+\p, Kürze \c, 14-15 FV (Splinter)}
\end{description}

% -------------  1C 1P -------------

\subsection{Weiterreizung nach 1\c1\p} \label{subsec:1c 1p}
TODO

% -------------  1C 1SA -------------

\subsection{Weiterreizung nach 1\c1SA} \label{subsec:1c 1SA}
TODO

% -------------  1C 2T* -------------

\subsection{Weiterreizung nach 1\c2\t} \label{subsec:1c 2t}
TODO

% -------------  1C 2C* -------------

\subsection{Weiterreizung nach 1\c2\c} \label{subsec:1c 2c}
{Mindestens 20 gemeinsame FV sind schon sicher. Mit etwas Zusatzstärke oder einer zweiten Farbe kann der Eröffner mit einem \textbf{forcierenden Versuchsgebot} einladen, auch Schlemmreizung kann über Versuchsgebote gestartet werden:}

\textcolor{cyan}{\begin{description}
    \item{2\p: Versuchsgebot mit üblicherweise 4+\p}
    \item{2SA*: Allgemeines Versuchsgebot mit Zusatzstärke}
    \item{3\t/3\k: Versuchsgebot mit üblicherweise 4+\tl bzw. 4+\k}
\end{description}}

\textcolor{cyan}{Partner kann mit Maximum oder mit einer zur zweiten Farbe passenden Hand mit 4\c annehmen und sonst mit 3\c ablehnen.}

% -------------  1C 2P -------------

\subsection{Weiterreizung nach 1\c2\p} \label{subsec:1c 2p}
TODO

% -------------  1C 2SA* -------------

\subsection{Weiterreizung nach 1\c 2SA} \label{subsec:1c 2SA}
\textcolor{cyan}{Falls vorhanden ist die Priorät eine schöne 5er-UF zu zeigen, ansonten Kürzen zeigen (sofern keine absolute Minimumhand).
\begin{description}
    \item{3\t*: Maximal 13 FV, kein Schlemminteresse}
    \item{3\c*: ab 14 FV \ra \textcolor{red}{3M-Modul s. \ref{sec:3M}}}
    \item{3\k*: ab 16 FV, \k -Kürze}
    \item{3\p*: ab 16 FV, \p -Kürze}
    \item{3SA*: ab 16 FV, \t -Kürze (!)}
    \item{4\t: zusätzlich 5+\t}
    \item{4\k: zusätzlich 5+\k}
    \item{4\c: kein Schlemminteresse, maximal eine Keycard ohne Trumpfdame}
\end{description}}

% -------------  1C 3T* -------------

\subsection{Weiterreizung nach 1\c 3\t} \label{subsec:1c 3t}
Partner hat 4+\c mit 9-11 FV oder 14-15 FV gezeigt:
\begin{description}
    \item{3\k*: Einladend zu Vollspiel oder Schlemm}
    \begin{description}
        \item{3\c: Minimum schwache Variante}
        \item{4\c: Maximum schwache Variante}
        \item{3SA*: Starke Variante ohne Kürze}
        \item{3\p*/4\t/4\k: Starke Variante mit Kürze \p/\t/\k}
    \end{description}
    \item{3\c: To play (falls Partner schwache Variante hatte), mit starker Variante korrigiert Partner in 4\c}
    \item{4\c: To play (auch falls Partner starke Variante hatte)}
\end{description}
\textcolor{cyan}{TODO: Was sind restliche Gebote?}

% -------------  1C 3K* -------------

\subsection{Weiterreizung nach 1\c 3\k} \label{subsec:1c 3k}
Partner hat 4+\c mit 12-13 FV gezeigt:
\begin{description}
    \item{3\c: To play}
    \item{4\c: To play}
    \item{3\p*: Schlemminteresse, fordert Partner zu Cue-Bids auf \ra 3M-Modul s. \ref{sec:3M}}
    \item{3SA*: Schlemminteresse, \p-Kürze (!)}
    \item{4\t/4\k: Schlemminteresse, \t/\k-Kürze}
\end{description}

% -------------  Antworten auf 1SA -------------

\section{Antworten auf 1SA} \label{sec:Antworten auf 1SA}
TODO

% -------------  Antworten auf 2T  -------------

\section{Antworten auf 2\t*} \label{sec:Antworten auf 2T}

\begin{description}
    \item{2\k* \gf Relay
               \tabto{12mm} a. 0-7 FL \newline
               \tabto{12mm} b. 8+ FL keine 5er-OF, keine 6er-UF}
    \item{2\c \gf 8+; 5+\c, forcierend bis 4SA}
    \item{2\p \gf 8+; 5+\p, forcierend bis 4SA}
    \item{2SA* \gf 8+; 6+\t, forcierend bis 4SA \ra 3\tl bestätigt Fit, Rest natürlich}
    \item{3\t* \gf 8+; 6+\k, forcierend bis 4SA \ra 3\kl bestätigt Fit, Rest natürlich}
    \item{alle weiteren Antworten zeigen 5+5+Verteilungen mit 8+, forcierend bis 4SA}
\end{description}

\section{Antworten auf 2\k*} \label{sec:Antworten auf 2K}

\section{Antworten auf 2\c*} \label{sec:Antworten auf 2C}

\section{Antworten auf 2\p*} \label{sec:Antworten auf 2P}

\section{Antworten auf 2SA} \label{sec:Antworten auf 2SA}

\endinput