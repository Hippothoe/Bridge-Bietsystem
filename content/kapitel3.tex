\chapter{Schlemmreizung}

\section{3M-Modul} \label{sec:3M}
\textcolor{red}{Erreicht die ungestörte Reizung die Stufe 3\cl bzw. 3\pl mit Schlemminteresse, kann mit dem 3M-Modul weitergereizt werden (M für Major bzw. Oberfarbe).}

TODO Non-Serious 3NT, Kontrollgebote

\section{Last Train}
Steht die Trumpffarbe fest und befindet die Auktion sich in Kontrollgeboten (ausgelöst bspw. durch Splinter oder ein erstes Cue-bid) zeigt das letzte Gebot vor 4 in Trumpffarbe eine nur einladende Hand zum Schlemm, falls es das letzte verbliebene Gebot ist. Insbesondere verspricht es nichts über die gereizte Farbe, wohl aber Kontrollen in vom Partner bereits gezeigten Schwächen.

Eine Hand ohne Schlemminteresse kann dagegen direkt abschließen. Eine Hand mit starkem Schlemminteresse und entsprechenden Kontrollen kann direkt zur Assfrage übergehen.

Für die Kontrolle in der Last Train-Farbe könnten wir über Lackwood sprechen (?). Aber das sollte man mit unseren 2-1 zusammen überlegen.

\section{Assfrage}
Nach gefundenem Oberfarbenfit (eventuell auch UF) ist 4SA die Assfrage. Diese fragt nach Anzahl an Keycards (Asse \& Trumpfkönig). Antworten über \textbf{1430}:
\begin{description}
    \item{passe: verboten}
    \item{5\t: 1 oder 4 Keycards}
    \item{5\k: 0 oder 3 Keycards}
    \item{5\c: 2 Keycards ohne Trumpfdame}
    \item{5\p: 2 Keycards mit Trumpfdame}
\end{description}
Folgegebote des Fragenden:
\begin{description}
    \item{Gebot des Fragenden in eigener Trumpffarbe: immer Abschluss}
\end{description}
Rest TODO

\subsection{Conditional RKCB}
Nach gefundenem Unterfarbenfit in m ist 4m die Assfrage für einen möglichen Schlemm. Partner kann mit dem nächsten Gebot ablehnen, die zwei weiteren zeigen dann Interesse und 14-30. Nach Ablehnung sind alle 4M Gebote to play, falls 
\begin{itemize}
	\item in M ein 7er-Fit besteht,
	\item in einer Hand 5+M gezeigt wurden oder
	\item in einer Hand 4+M gezeigt wurden und die andere Oberfarbe uninteressant ist.
\end{itemize}

\endinput