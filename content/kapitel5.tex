\chapter{Gegenreizung}

\section{Gegen 1 in Farbe}

\subsection{Überruf 1 in Farbe}
(1\t/1\k/1\c) 1\k/1\c/1\p: mindestens eine Figur in der Farbe, mindestens 5er-Länge, mindestens eine Figur in der Farbe, ab 9 FL

\subsection{Überruf 1SA}

\subsection{Takeout-Kontra}
Kontra gegen eine Eröffnung von 1 in Farbe ist in erster Linie ein Takeout-Kontra und zeigt eigene Eröffnungsstärke (12+ F) sowie mindestens 3er-Länge in allen Restfarben.

Idealerweise hat man 4er-Länge in mindestens einer ungereizten OF und eine Kürze in der vom Gegner eröffneten Farbe (maximal Double). Je mehr Punkte man hat, desto mehr kann die Verteilung von einem idealen Takout-Kontra abweichen.

TODO Antworten

\subsection{Stärke-Kontra}
In der Gegenreizung gegen 1 in Farbe reizt man beliebige Hände ab 18 FL zunächst mit X, als wäre es ein Takeout-Kontra. Anschließend gibt man in jeden Fall ein Wiedergebot ab, um sein Stärkekontra zu offenbaren und seine Hand weiter zu beschreiben. (Dies gilt nur unterhalb von Vollspielniveau, Vollspiele dürfen natürlich gepasst werden.)

TODO Bedeutung von Wiedergeboten

\subsection{Michaels}

\subsection{Unusual 2NT}

\subsection{Sperrgebote}

\section{Gegen 1SA}
\textcolor{cyan}{Gegenreizung gegen 1SA mit \textbf{Multi-Landy}. Die Stärke ist abhängig von der Gefahrenlage: In Nichtgefahr gegen Gefahr ab 8 FL, sonst ab 10 FL.}

\textcolor{cyan}{\begin{itemize}[label={}]
\item{X: ab 14 FL, Strafkontra}
\item{2\t*: mindestens 5-4 in OF \ra s. \ref{G 1SA 2t}}
\item{2\k*: OF-Einfärber, üblicherweise 6+ (mit 5332-Verteilung ist häufig besser gegenzuspielen) \ra s. \ref{G 1SA 2k}}
\item{2\c*/2\p*: 5er\c/\pl + 4er-UF \ra s. \ref{G 1SA 2c/p}}
\item{2SA*: 5+\kl 5+\tl}
\item{3\t/3\k/3\c/3\p: üblicherweise 7er-Farbe, Sperrgebot}
\end{itemize}}

\subsection{Gegenreizung 1SA 2\t} \label{G 1SA 2t}
eine länger: zeigen, beide gleichlang: 2 Karo; alles to play.

Stärker: 2SA, fragt nach längerer Farbe

\subsection{Gegenreizung 1SA 2\k} \label{G 1SA 2k}
TODO

\subsection{Gegenreizung 1SA 2\c/2\p} \label{G 1SA 2c/p}
TODO


\section{Gegen Sperrgebote}

\subsection{Gegen Sperrung auf 2er-Stufe}
Die Gegenreizung bietet verschiedene Optionen.\footnote{\url{https://www.bridgebum.com/lebensohl_over_weak_two.php}}\footnote{\url{https://kwbridge.com/preempts.htm}}
\begin{description}
    \item Direktes Gebot auf 2er-Stufe: 10-17 FL, 5er Farbe. Der gebotene Kontrakt sollte mit optimistischen \mat{7} Punkten beim Partner erfüllbar sein.
    \item Kontra X: 12+ FL, Kürze in Gegnerfarbe, 3-4 in allen ungebotenen Farben. \mat{Oder: 18+FL und sehr lange Farbe}
    \item 2NT: 14+FL, ausgeglichene Hand mit Stoppern
    \item Überruf: \textcolor{brown}{in UF: stark und beide OF zu fünft; in OF: stark und ausgeglichen, fragt nach Stopper in Gegnerfarbe für 3NT. Partner verneint mit natürlichen Geboten.}
\end{description}

Nach Kontra spielen wir \textcolor{brown}{Lebensohl}:
\begin{description}
    \item 2NT*: schwache Hand, forciert ein 3\t* Relay (ausgenommen starkes Kontra). Jede Antwort auf das Relay ist abschließend .
    \item Direktes Gebot in Farbe: einladende Hand und 4er Farbe
\end{description}
Für partieforcierende Hände gibt es drei \mat{Optionen}:
\begin{description}
    \item 3NT: keine (ungebotene) OF zu viert, kein Stopper in Gegnerfarbe (also sehr sehr stark). Partner passt mit Stopper. Antworten auf 4er Stufe sind Schlemminteresse. Die 5er Stufe schließt ab.
    \item 2NT* - 3\t* - 3NT: keine (ungebotene) OF zu viert aber \textit{slow shows} Stopper in Gegnerfarbe  
    \item 2NT* - 3\t* - Überruf: mindestens eine ungebotene OF zu viert und \textit{slow shows} Stopper in Gegnerfarbe  
\end{description}


\section{Sonstiges}

\subsection{Wiederbelebungskontra}

\endinput