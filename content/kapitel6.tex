\chapter{Gegenspiel}

\section{Gegen Farbkontrakte}
\subsection{Ausspiel}
\begin{itemize}
\item{Höchste Karte von Sequenzen}
\item{Ansonsten 3./5. höchste Karte}
\item{Hoch von einem Double \color{brown} (aber: bei Überstechen des Gegners immer niedrig von Sequenz!)}
\end{itemize}

\subsection{Markierungen}
Markierungen nur, wenn nicht um den Stich gekämpft wird und nur mit Karten, die man entbehren kann.

Bei Ausspiel vom Partner:
\begin{itemize}
    \item Attitude (niedrig positiv)
    \item Bei Single am Dummy stattdessen Lavinthal
\end{itemize}

Bei Ausspiel vom Gegner: Längenmarken (niedrig-hoch gerade, hoch-niedrig ungerade) \textcolor{brown}{ gezählt werden dabei die Anzahl der Karten in der Hand zu Beginn.}

{Beim Abwurf generell direkte Zu- und Abmarken: Funktioniert wie Attitude (niedrig: positiv)} \textcolor{red}{(kein Lavinthal!)}


Wenn man Partner einen Schnapper gibt, markiert man mit der ausgespielten Karte Lavinthal (um ein erfolgreiches Rückspiel zu finden, mit dem Ziel einen weiteren Schnapper zu erreichen)

\textcolor{red}{1. Stich: Weber-Marke \ra Wenn man nicht um den Stich kämpft, signalisiert eine mittlere Karte Interesse an der ausgespielten Farbe und eine niedrige bzw. hohe Karte Interesse an der niedrigeren bzw. höheren Restfarbe. Stärkste Zumarke ist die 6. Konkrete Reihenfolge: 6-5-7-4-8-3-9-2-10}

\section{Gegen SA-Kontrakte}
\subsection{Ausspiel}

\begin{itemize}
\item{Höchste Karte von Sequenzen}
\item{Ansonsten 2./4. höchste Karte (In Partnerfarbe 3./5.)}
\item{TODO von einem Double}
\end{itemize}

\subsection{Markierungen}
Markierungen nur, wenn nicht um den Stich gekämpft wird und nur mit Karten, die man entbehren kann

Bei Ausspiel vom Partner: Attitude (niedrig postiv)

Bei Ausspiel vom Gegner: Längenmarken (niedrig-hoch gerade, hoch-niedrig ungerade)

{Beim Abwurf generell direkte Zu- und Abmarken: Funktioniert wie Attitude (niedrig: positiv)} \textcolor{red}{(kein Lavinthal!)}

\endinput